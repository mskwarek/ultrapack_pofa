\section*{Zadanie 5.}
\begin{task}
Jakie są własności pół promieniowania dipola Hertza. Jak wykorzystuje się wzory na te pola w projektowaniu anten? Co to jest zysk kierunkowy anteny i jakie są metody jego powiększania? Zilustrować te metody na praktycznych przykładach kilku typów anten.\\
\end{task}

\begin{solution}

$$H\approx\vec{i_{\varphi}}\cfrac{I_{0}\beta l}{4\pi r}\sin{\theta}e^{j(\omega t-\beta r)}$$
$$E\approx\vec{i_{\theta}}\cfrac{I_{0}\beta^{2} l}{4\pi\epsilon\omega r}\sin{\theta}e^{j(\omega t-\beta r)}$$
Wyprowadzone między innymi z równań Maxwella.\\
Analizując powyższe wektory dowiadujemy się że:
\begin{itemize}
\item w dużej odległości od dipola Hertza, pole elektryczne i magnetyczne są do siebie prostopadłe, fazy są zgodne, a stosunek amplitud równy impedancji właściwej ośrodka.
\item wektor Poyntinga wynikający z istnienia pól jest zawsze skierowany wzdłuż wersora $\vec{i_{r}}$ co w połączeniu z wcześniejszymi cechami oznacza, że dipol Hertza jest źródłem promieniowania energii elektromagnetycznej.
\item pole elektyczne i magnetyczne w postaci wzorów opisują bezpośrednio gęstość mocy promieniowania dlatego nazwano je polami promieniowania
\item w dostatecznie malej odległości od dipola możemy przyjąć, energia pola magnetycznego jest pomijalnie mała w stosunku do energii pola elektrycznego.
\end{itemize}
\textbf{Zysk kierunkowy anteny}\\
Jest to stosunek mocy, jaką musiałaby wypromieniować izotropowa antena porównawcza, do mocy jaką promieniuje antena rozważana, jeśli gęstość mocy w tym samym kierunku ($\varphi,\ \theta$) i tej samej odległości są sobie równe.
$$G(\varphi,\ \theta)=\cfrac{p_i}{p}$$
\textbf{Związek parametrów anten z polami}\\
Niektóre własności oraz parametry można znaleźć dzięki wykorzystaniu wzorów na pole elektryczne oraz magnetyczne, np. do znalezienia charakterystyki promieniowania należy znaleźć najpierw pole magnetyczne w punkcie $P(r, \ \varphi,\ \theta)$ w strefie dalekiej od anteny. Pole to jest sumą pól dipoli elementarnych, z których składa się antena.\\
Zauważamy, że charakterystyką promieniowania możemy sterować za pomocą pola magnetycznego. Podobnie jest z innymi parametrami np. z rezystancją promieniowania, która wiąże się z mocą promieniowania obliczanej z wektora Poyntinga.
$$S(r,\ \varphi,\ \theta)=\cfrac{1}{2}E_{\theta}H_{\varphi}^{*}$$
Parametry, których wzory powiązane są ze wzorami pól mają wielką rolę dla projektujących anteny.\\

\textbf{Do uzyskania dużej kierunkowości stosujemy:}\\
\begin{itemize}
\item 'Ściany' antenowe a nawet prestrzenne rozmieszczenie dipoli
\item elektroniczne sterowanie amplitudy i fazy prądu
\item anteny z reflektorem
\item układy promieniujących szczelin w ściance falowodu
\end{itemize}

\end{solution}