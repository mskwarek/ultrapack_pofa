\section*{Zadanie 25.}
\begin{task}
Falowód prostokątny o bokach $a = 5 cm$ i $b = 2.5 cm$ jest wypełniony bezstratnym dielektrykiem o poszukiwanej względen przenikalności elektrycznej $\varepsilon_w$. Stwierdzono, że dla częstotliwości dwa razy wiekszej od częstotliwości granicznej długość fali w falowodzie jest równa długości fali w próżni. Obliczyć $\varepsilon_w$. Który rodzaj fali e.m. rozchodzi się w falowodzie (uazasadnić) ?
\end{task}

\begin{solution}

$f = 2 \cdot f_g$ $\qquad$
$\lambda_{w \ falowodzie} = \lambda_{w \ prozni}$ \\
$\lambda_{w \ falowodzie} = \frac{2 \pi}{\beta_z} = \frac{\frac{2 \pi}{\beta}}{\sqrt{1-(\frac{\beta_g}{\beta})^2}} =  \frac{\lambda}{\sqrt{1-(\frac{\lambda}{\lambda_g})^2}} = \frac{\lambda}{\sqrt{1-(\frac{\omega_g}{\omega})^2}} = \frac{\lambda}{\sqrt{1-(\frac{f_g}{f})^2}}$ \\
$\lambda_{w \ falowodzie} =  \frac{\lambda}{\sqrt{1-(\frac{f_g}{f})^2}} \stackrel{\substack{f = 2 \cdot f_g}}{=} \frac{\lambda}{\sqrt{1-(\frac{\not f_g}{2 \not f_g})^2}} = \frac{\lambda}{\sqrt{1-(\frac{1}{2})^2}} = \frac{\lambda}{\sqrt{1-\frac{1}{4}}} = \frac{\lambda}{\sqrt{\frac{3}{4}}}$ \\ \\
$\lambda_{w \ prozni} = \frac{c}{f}$ \\ \\
$\lambda = \frac{c}{\sqrt{\varepsilon_w} \cdot f}$ \\ \\
$\lambda_{w \ falowodzie} = \lambda_{w \ prozni} \qquad \Rightarrow \qquad  \frac{\frac{\not c}{\sqrt{\varepsilon_w} \cdot \not f}}{\sqrt{\frac{3}{4}}} = \frac{\not c}{\not f} \qquad \Rightarrow \qquad \frac{\frac{1}{\sqrt{\varepsilon_w} }}{\sqrt{\frac{3}{4}}} = 1 \qquad \Rightarrow \qquad \frac{\frac{1}{\varepsilon_w }}{\frac{3}{4}} = 1 = \qquad \Rightarrow \qquad \varepsilon_w  = {\frac{4}{3}} $ \\ \\

\textit {Nie da się okreslić rodzaju fali e.m. jaka rozchodzi sie w falowodzie, jest za mało danych. Wymiary falowodu były podane "dla zmyłki"}  dr J.Piotrowski 2014


\end{solution}