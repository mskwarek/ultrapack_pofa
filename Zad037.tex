\section*{Zadanie 37}
\begin{task}
Jaki jest zakres stosowalności równań Maxwella. Podać przykłady praktycznych problemów elektroniki które mogą oraz tych które nie mogą być rozwiązane za ich pomocą. Czy wszystkie równania Maxwella mają równie szerokie zastosowania (to znaczy w jakich okolicznościach niektóre z równań są zależne od innych) \\
\end{task}

\begin{solution}
Równania Maxwella - cztery podstawowe równania elektrodynamiki klasycznej zebrane i rozwinięte przez Jamesa Clerka Maxwella. Opisują one właściwości pola elektrycznego i magnetycznego oraz zależności między tymi polami. Z równań Maxwella można wyprowadzić między innymi równania falowe fali elektromagnetycznej oraz wyznaczyć prędkość takiej fali propagującej (rozchodzącej się) w próżni (prędkość światła).


Równania Maxwella w postaci ogólnej 
$$ \nabla \times \vec{E} = -  \frac{ \delta  \vec{B} }{\delta t}  $$
 $$\nabla \times \vec{H} =  \vec{J} + \frac{ \delta  \vec{D} }{\delta t} $$
$$\nabla  \cdot \vec{D} =  \rho $$
$$\nabla  \cdot \vec{E} =  0  $$

Równania Maxwella w postaci całkowej \\
 $$\oint\limits_{l}\vec{E}dl = - \iint\limits_{S}\frac{\delta \vec{B}}{\delta t}$$
 $$\oint\limits_{l}\vec{H}dl =  \iint\limits_{S}( \vec{J} + \frac{\delta \vec{B}}{\delta t} )  $$
$$\oiint\limits_{S} \vec{D} \vec{n} ds = \iiint\limits_{V} \rho dv $$
$$\oiint\limits_{S} \vec{B} \vec{n} ds = 0 $$

Równania Maxwella w postaci zespolonej \\
$$\nabla \times \vec{E} = -j \omega \vec{B} $$
$$\nabla \times \vec{H} = \vec{J} + j\omega\vec{D} $$
$$\nabla \cdot \vec{D} = \rho $$
$$\nabla \cdot \vec{B} = 0 $$


Równania Maxwella tworzą podstawę elektrodynamiki klasycznej oraz podstawę do wyprowadzenia prawa Biota-Savarta. Całkując po czasie I r.M. i biorąc dywergencję IV r.M. otrzymujemy równanie opisujące zasadę zachowania ładunku
$$\nabla \cdot J = -  \frac{\delta \rho}{\delta t} $$

Całkująć II r.M. po powierzchni s której brzegiem L jest obwód elektryczny, dostrzegamy że gdy strumień pola magnetycznego przez s nie zmienia się to II r.M. przyjmuje postac 

$$\int\limits_L \vec{E} dL = 0 $$

i staje się napięciowym prawem Kirhoffa

$$ \int\limits_{a}^{b} \vec{E} dL + \int\limits_{b}^{c}\vec{E} dL + \int\limits_{c}^{d} \vec{E} dL + \int\limits_{d}^{a} \vec{E} dL = 0\\  U_{ab} + U_{bc} + U_{cd} + U_{da} = 0\\ \sum_{i=a}^{d} U_{i}=0  $$

Szczególnie istotne jest II r.M. które jest rozwinięciem Prawa Amprere'a - prawa wiążącego indukcję magnetyczną wokół przewodnika z prądem z natężeniem prądu elektrycznego przepływającego w tym przewodniku. Mówi ono że przepływający w przewodniku prąd wytwarza wokół niego pole magnetyczne. 

$$\nabla \times \vec{H} =  \vec{J} + \frac{ \delta  \vec{D} }{\delta t} $$

Maxwell rozwinął to prawo do postaci polowej, dodając do wyrażenie na prąd przesunięcia

$$J_{D} = \frac{\delta \vec{D}}{\delta t} $$

w wyniku obserwacji że pole magnetyczne tworzy nie tylko przepływający prąd, ale również zmienne pole elektryczne wokół przewodnka (prad przesunięcia).

Dużą rolę ma również I.r.M. wynikające z prawa Faraday'a mówiące o przeciwdziałaniu polu przez je samo, tj. zmienne pole magnetyczne wokół przewodnika indukuje w nim ruch ładunków wytwarzający prąd elektryczny co do wartości równy szybkości zmiany pola magnetycznego o kierunku takim, że indukowane przez niego pole magnetyczne przeciwdziała polu które ruch ładunków wytworzyło.

Jak łatwo zauważyć analizując wzory, wzajemnie zalezne od siebie są I i II oraz III i IV r.M. 
\\

\end{solution}