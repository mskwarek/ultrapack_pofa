\section*{Zadanie 20.}
\begin{task}
Zapisać w postaci rzeczywistej wektor pola elektrycznego i magnetycznego fali płaskiej o polaryzacji kołowej prawoskrętnej, rozchodzącej się w próżni w kierunku wersora \textbf{k} o składowych $\big{(}-\cfrac{1}{2},0,\cfrac{\sqrt{3}}{2}  \big{)}$ jeśli wiadomo, że w chwili t=0, w środku układu współrzędnych pole magnetyczne jest równoległe do osi Oy.\\
\end{task}

\begin{solution}


$$\vec{k}=\big{[}-\cfrac{1}{2};0;\cfrac{\sqrt{3}}{2}\big{]} $$
Z własności rozchodzenia się fali płaskiej wynika, że: $\vec{k}\cdot\vec{H}=0$, więc:
$$\vec{k}\cdot\vec{H}=-\cfrac{1}{2}H_{x}+\cfrac{\sqrt{3}}{2}H_{z}=0 \ \ \implies \ \ H_{x}=\sqrt{3}H_{z}$$
Wiemy z polecenia, że polaryzacja fali ma być kołowa, co mówi nam, że składowe pola muszą być przesunięte o $\cfrac{\pi}{2}$ względem siebie. Jeżeli w punkcie (0,0,0) mamy amplitudę rzeczywistą, to druga składowa musi być urojona, a więc pole $H$ można zapisać jako: $\vec{H}=\vec{H_{1}}+\vec{H_{2}}$. Z informacji, że mamy falę spolaryzowaną kołowo wnioskujemy, że amplitudy obu składowych są sobie równe:
$$|\vec{H_{1}}|=|\vec{H_{2}}| \ \ \implies \ \ H_{y}=\sqrt{H_{x}^{2}+H_{z}^{2}}=\sqrt{3H_{z}^{2}+H_{z}^{2}}=2H_{z}$$
$$H_{y}=\stackrel{+}{-}2H_{z}\ \ \ H_{x}=\stackrel{+}{-}2\sqrt{3}H_{z}$$
Wynika z tego, że zadania wymaga rozpatrzenia dwóch przypadków:\\
\begin{enumerate}[1*]
\item $\vec{H^{'}} = H_{z} e^{j(\omega t - \beta \vec{k} \cdot\vec{r})}\big{(} 2\sqrt{3}j\vec{i_{x}} + 2\vec{i_{y}} +j\vec{i_{z}} \big{)}$
\item $\vec{H^{''}} = H_{z} e^{j(\omega t - \beta \vec{k} \cdot\vec{r})}\big{(} -2\sqrt{3}j\vec{i_{x}} - 2\vec{i_{y}} +j\vec{i_{z}} \big{)}$
\end{enumerate}
Korzystamy z własności fali płaskiej:
$$ \vec{E} = \cfrac{\vec{H}\times\vec{k}}{Z} $$\\

\begin{enumerate}[1*]
\item $\vec{E_{1}}=\cfrac{\vec{H^{'}}\times\vec{k}}{Z} = \cfrac{H_{z}}{Z} e^{j(\omega t - \beta\vec{k} \cdot\vec{r})}          \begin{vmatrix}				                                                             \vec{i_{x}}&\vec{i_{y}}&\vec{i_{z}}\\
					                    2\sqrt{3}j&2&j\\
					                    -\cfrac{1}{2}&0&\cfrac{\sqrt{3}}{2}\end{vmatrix} = \cfrac{H_{z}}{Z} e^{j(\omega t - \beta\vec{k} \cdot\vec{r})}\big{(}\sqrt{3}\vec{i_{x}}+\cfrac{5}{2}j\vec{i_{y}} +\vec{i_{z}}\big{)}\\
\vec{E_{1}}=\cfrac{H_{z}}{Z} e^{j(\omega t - \beta\vec{k}\cdot\vec{r})}\big{(}\sqrt{3}\vec{i_{x}}+\vec{i_{z}}\big{)} + \cfrac{H_{z}}{Z} e^{j(\omega t - \beta\vec{k}\cdot\vec{r}+\cfrac{\pi}{2})}\cfrac{5}{2}\vec{i_{y}}\\
\vec{E_{1}}_{re} =\cfrac{H_{z}}{Z}\big{(} (\sqrt{3}\vec{i_{x}}+\vec{i_{z}})\cos{(\omega t - \beta\vec{k}\cdot\vec{r})} - \cfrac{5}{2}\vec{i_{y}}\sin{(\omega t - \beta\vec{k}\cdot\vec{r})}  \big{)} $
					      
					                    
\item $\vec{E_{2}}=\cfrac{\vec{H^{''}}\times\vec{k}}{Z} =\cfrac{H_{z}}{Z} e^{j(\omega t - \beta\vec{k} \cdot\vec{r})}           \begin{vmatrix}
					                    \vec{i_{x}}&\vec{i_{y}}&\vec{i_{z}}\\
					                    -2\sqrt{3}j&-2&j\\
					                    -\cfrac{1}{2}&0&\cfrac{\sqrt{3}}{2}\end{vmatrix} = \cfrac{H_{z}}{Z} e^{j(\omega t - \beta\vec{k} \cdot\vec{r})}\big{(}-\sqrt{3}\vec{i_{x}}+\cfrac{5}{2}j\vec{i_{y}} -\vec{i_{z}}\big{)}\\
\vec{E_{2}}=\cfrac{H_{z}}{Z} e^{j(\omega t - \beta\vec{k}\cdot\vec{r})}\big{(}-\sqrt{3}\vec{i_{x}}-\vec{i_{z}}\big{)} + \cfrac{H_{z}}{Z} e^{j(\omega t - \beta\vec{k}\cdot\vec{r}+\cfrac{\pi}{2})}\cfrac{5}{2}\vec{i_{y}}\\
\vec{E_{2}}_{re} =\cfrac{H_{z}}{Z}\big{(} -(\sqrt{3}\vec{i_{x}}+\vec{i_{z}})\cos{(\omega t - \beta\vec{k}\cdot\vec{r})} - \cfrac{5}{2}\vec{i_{y}}\sin{(\omega t - \beta\vec{k}\cdot\vec{r})}  \big{)}					                    
					                   $\\
\end{enumerate}
Polaryzację prawoskrętną mogą opisywać dwa wyrażenia:
$$ \vec{E} = \vec{i_{x}}E_{r}cos{(\omega t)} + \vec{i_{y}}E_{r}sin{(\omega t)} \ \ \ \vee\ \ \ 
    \vec{E} = -\vec{i_{x}}E_{r}sin{(\omega t)} + \vec{i_{y}}E_{r}cos{(\omega t)}$$\\
W naszym zadaniu przypadek 2* pasuje do pierwszego wyrażenia (wystarczy wyłączyć - przed nawias).\\
Więc ostatecznie odpowiedź do zadania:\\
$ \vec{H}_{re} = H_{z}\big{(} (-2\sqrt{3}\vec{i_{x}}+ \vec{i_{z}})\sin{(\omega t - \beta \vec{k} \cdot\vec{r})} - 2\vec{i_{y}}\cos{(\omega t - \beta \vec{k} \cdot\vec{r})} \big{)} $ 

$ \vec{E}_{re} =\cfrac{H_{z}}{Z}\big{(} -(\sqrt{3}\vec{i_{x}}+\vec{i_{z}})\cos{(\omega t - \beta\vec{k}\cdot\vec{r})} - \cfrac{5}{2}\vec{i_{y}}\sin{(\omega t - \beta\vec{k}\cdot\vec{r})}  \big{)}	$
					                   




\end{solution}