\section*{Zadanie 27.}
\begin{task}
Obliczyć długość fali elektromagnetycznej o częstotliwości f=30 MHz
\begin{enumerate}[a)]
\item w dielektryku o $\epsilon_{w}=9$, $\mu_{w}=1$, $\sigma=0$
\item w przewodniku o $\epsilon_{w}=1$, $\mu_{w}=1$, $\sigma=10^{7} \ \cfrac{S}{m}$
\item w plaźmie o częstotliwości własnej plazmy $f_{p}=20\ MHz$
\item w plaźmie o częstotliwości drgań własnych $f_{p}=50\ MHz$
\end{enumerate}
Uwaga: $\mu_{0}=4\pi10^{-7} \cfrac{H}{m}$\\
Krótko omówić fizyczne interpretacje otrzymanych wyników.\\
\end{task}

\begin{solution}
\begin{enumerate}[a)]
\item $\lambda=\cfrac{1}{f\sqrt{\mu\epsilon}}=\cfrac{c}{f\sqrt{\mu_{w}\epsilon_{w}}}=\cfrac{10}{3} \ m$
\item $v=\cfrac{\omega}{\beta}$ $\lambda=
            \cfrac{2\pi}{\beta}=\cfrac{2\pi}{\sqrt{\cfrac{\omega\mu\sigma}{2}}}\\ \lambda=\cfrac{1}{\sqrt{30}}*10^{-3}\ m$
\item $\epsilon_{p}=\epsilon_{0}[1-(\cfrac{\omega_p}{\omega})^{2}]=\epsilon_{0}[1-(\cfrac{f_{p}}{f})^{2}]$\\
                $\lambda_{p}=\cfrac{\lambda_{0}}{\sqrt{\epsilon_{p}}}=\cfrac{c}{f\sqrt{\epsilon_{0}[1-(\cfrac{f_{p}}{f})^{2}]}}$
\item Fala się nie rozchodzi, bo $\omega<\omega_{p}$

\end{enumerate}
\end{solution}


