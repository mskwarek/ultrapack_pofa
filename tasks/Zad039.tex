\section*{Zadanie 39.}
\begin{task}
Jakie są właściwości elektryczne zimnej plazmy? Gdzie występuje ona w przyrodzie? Jaki ma wpływ na propagację fal radiowcyh w atmosferze? ( z wyróżnieniem własności fal długich, krótkich, UKF i mikrofal) ?\\
\end{task}


\begin{solution}
Plazma – zjonizowana materia o stanie skupienia przypominającym gaz, w którym znaczna część cząstek jest naładowana elektrycznie. Mimo że plazma zawiera swobodne cząstki naładowane, to w skali makroskopowej jest elektrycznie obojętna.
Plazma zimna powstaje przy odpowiednio niskich temperaturach i gęstościach, w warunkach ziemskich (na przykład podczas wyładowań atmosferycznych - znajduje się ona zatem w jonosferze) i w zbudowanych przez człowieka urządzeniach (na przykład plazmotronach). W jej skład, prócz składników tworzących plazmę gorącą, wchodzić mogą również atomy i ich jony, a także cząsteczki (zarówno obojętne, jak i zjonizowane).\\
Warunki w plazmie.\\
\begin{textbf} {1.}\end{textbf} Fale o $\omega < \omega_{p}$ są tłumione w plazmie. W przypadku padania fali o takiej omedze z próżni na warstwę jonosfery ulegnie ona całkowitemu odbiciu.\\
\begin{textbf} {2.}\end{textbf} Fale i $\omega > \omega_{p}$ rozchodzą się w plazmie. W przypadku padania fali ukośnie z próżni na jonosferę fala załamania odchyla sie od normalnej(gdyż przechodzi do ośrodka rzadszego)\\
\begin{textbf} {3.}\end{textbf} Fale o $\omega >> \omega_{p}$ rozchodzą się w plazmie tak jak w próżni, ponieważ $\varepsilon_{p} \longrightarrow \varepsilon_{0}$. Tylko takie fale (w prakrtyce - mikrofale) swobodnie przechodzą przez jonosferę i mogą być użyte w komunikacji satelitarnej.\\
PRZYPOMNIENIE: Im mniejsza długość fali, tym większa jej częstotliwość/energia/pulsacja drgań własnych\\
Fale długie (od 100m do 2000m) rozchodzą się nisko po powierzchni ziemi, zimna plazma zawarta w jonosferze nie wpływa na ich propagację.\\
Fale krótkie (od 10m do 100m) mają własność odbijania się od górnych warstw atmosfery (m.in. jonosfery). Plazma zamawrta w jonosferze sprawia że zgodnie z warunkiem \begin{textbf} {(1.)}\end{textbf} fale te odbijają się od jonosfery co umożliwia im osiągnąć globalny zasięg.\\
UKF - "Ultra Krótkie Fale" (od 10cm do 1m) - ich pulsacja jest większa od pulsacji własnej plazmy, więc zgodnie z warunkiem \begin{textbf} {(2.)}\end{textbf}  fale w jonosferze "przechodzą dalej" (z załamaniem ) i nie następuje ich odbicie z powrotem do odbiorników/nadajników. Powoduje to że fale te mają jedynie zasięg jaki umozliwia rozsyłanie ich liniowo, stosowane są w radiostacjach.\\
Mikrofale (do 30cm) - dale te mają bardzo dużą częstotliwość przez co ich pulsacja drgań własnych jest duzo wieksza od tej wielkośći w plazmie. Zgodnie z warunkiem \begin{textbf} {(3.)}\end{textbf} fale przechodzą przez zimną plazmę zawartą w jonosferze jak przez próżnię (nie załamują się) co umożliwia użycie tych fal w astrologii (np. radioteleskopy).

\end{solution}