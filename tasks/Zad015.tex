\section*{Zadanie 15.}
\begin{task}
Porównać (w zwięźle sformułowanych punktach) ogólne własności fal TEM w liniach przesyłowych oraz \textsl{E}
i \textsl{H} w falowodach pod względem: (częstotliwości granicznej, definicji parametrów obwodowych, zmiennośći
parametrów w funkcji częstotliwości, prędkości fazowej i grupowej, wielorodzajowości).\\
\end{task}

\begin{solution}

\begin{itemize}
\item Częstotliwość graniczna:
    \begin{itemize}
    \item dla TEM 0 Hz w linii współosiowej
    \item dla falowodów zależy od rodzaju fali jaki się rozchodzi
        $$w_{g}=\sqrt{\cfrac{(\cfrac{m\pi}{a})^{2} + (\cfrac{n\pi}{b})^{2}}{\mu\epsilon}} $$
        \begin{itemize}
        \item $f<f_{g}$ - fala się nie rozchodzi (tłumiona)
        \item $f>f_{g}$ - fala się rozchodzi
        \end{itemize}
    \end{itemize}
\item parametry obwodowe
    \begin{itemize}
        \item TEM 
        \begin{itemize}
            \item maksymalne napięcie między przewodami:
                $U= \int_{b}^{a}E_{0}\cfrac{b}{\rho}d\rho$
            \item maksymalna wartość prądu w przewodach:
                $I=\oint_{l}\vec{H}d\vec{l}=\cfrac{\epsilon_{0}}{Z}2\pi b $
            \item impedancja charakterystyncza: 
                $Z_{c}=\cfrac{Z_{0}}{2\pi}\ln{\cfrac{a}{b}} $  
            \item pojemność jednostkowa:
                $C_{1}=\cfrac{q_{1}}{U}=\cfrac{2\pi\epsilon}{\ln{\cfrac{a}{b}}} $
            \item indukcyjność jednostkowa:
                $L_{1}=\cfrac{\mu}{2\pi}\ln{\cfrac{a}{b}} $
            \end{itemize}
          \item falowód 
            \begin{itemize} 
                \item impedancja charakterystyczna:
                $Z_{CUI}=\cfrac{U}{I}$$ $$Z_{CPU}=\cfrac{U^2}{2P}$, 
                $Z_{CPI}=\cfrac{2P}{I^2}$
                \item współczynnik propagacji: $\gamma_{z}=\sqrt{\beta_{g}^{2}-\beta^{2}}$
                \item $\beta_{z}=\beta\sqrt{1-\cfrac{\beta_{g}^{2}}{\beta}}$
                \item graniczna długość fali: $\lambda_{g}=\cfrac{2\pi}{\beta_{g}}=\cfrac{2}{\sqrt{(\cfrac{m}{a})^{2}+(\cfrac{n}{b})^{2}}} $
                \item długość fali w falowodzie:
                     $\lambda_{z}=\cfrac{2\pi}{\beta_{z}}=\cfrac{\lambda}{\sqrt{1-(\cfrac{\lambda}{\lambda_{g}})^{2}}} $            
                \item impedancja falowa dla fali typu E:
                     $Z_{f}=Z\sqrt{1-(\cfrac{\omega_{g}}{\omega})^{2}} $
                \item impedancja falowa dla fali typu H:
                    $Z_{f}=\cfrac{Z}{\sqrt{1-(\cfrac{\omega_{g}}{\omega})^{2}}} $
            \end{itemize} 
    \end{itemize}
\item zmienność parametrów w funkcji częstotliwości:
    \begin{itemize}
    \item TEM - od częstotliwości zależy współczynnik propagacji fali $\gamma=j\omega\sqrt{L_{1}C_{1}}$. 
            Wraz ze wzrostem częstotliwości wzrasta współczynnik propagacji.
    \item falowód - $\gamma_{z}$, $\beta_{z}$, $\lambda_z$, $Z_{f}$ zależą od częstotliwości      
    \end{itemize}
\item prędkość fazowa i grupowa
    \begin{itemize}
    \item dla fali tem TEM: $v_{g} = v_{f}$
    \item w falowodach prędkości są różne: $$v_{f}=\cfrac{V}{\sqrt{1-(\cfrac{\omega_{g}}{\omega})^2}} $$  
            $$v_{g}=v\sqrt{1-(\cfrac{\omega_{g}}{\omega})^{2}} $$
    \end{itemize}
\item wielorodzajowość
    \begin{itemize}
    \item w linii TEM zwykle jeden rodzaj, ale dla bardzo dużych częstotliwości można wzbudzić wyższe rodzaje
    \item w falowodzie  rodzaj zależy od wymiarów falowodu, można mieć jednocześnie kilka rodzajów. W falowodzie
            prostokątnym, wśród fal typu \textsl{E} podstawowym rodzajem jest $E_{11}$. Rodzaj ten ma największą
            wartość długości fali wśród rodzajów typu $E$.\\
            W zbiorze WSZYSTKICH rodzajów fal w falowodzie prostokątnym, podstawowym jest rodzaj $H_{10}$.    
    \end{itemize}
\end{itemize}

\end{solution}
