\section*{Zadanie 21.}
\begin{task}
Omówić własnośći propagacyjne w atmosferze fal elektromagnetycznych w zakresach częstotliwości zawartych między 200kHz a 100GHz. Wyjaśnić zjawiska fizyczne, decydujące o tych własnościach w poszczególnych zakresach oraz podać jak wpływają one na możliwość zastosowań w radiokomunikacji naziemnej i satelitarnej.\\
\end{task}

\begin{solution}
\begin{itemize}
\item Fale elektromagnetyczne emitowane są z nadajnika a odbierane przez odbiornik.
\item Fale elektromagnetyczne przechodzą przez izolatory a nie przechodzą przez przewodniki.
\item Podlegają zjawisku odbicia zgodnie z prawem odbicia.
\item Fala elektromagnetyczne to fala poprzeczna.
\item Fale elektromagnetyczne ulegają zjawisku dyfrakcji, interferencji, polaryzacji.\\
\end{itemize}

\textbf{Podział fal elektromagnetyczntch}
\begin{itemize}
\item długie (100 do 2000m) - rozchodzą się nisko po powierzchni ziemi, są słabo pochłaniane przez ziemię, dają dobry odbiór nawet na odległość kilku tysięcy kilometrów (stosowane do przesyłania informacji na duże odległości);
\item średnie (200 do 600m) - ulegają dużemu pochłanianiu przez ziemię, dają dobry odbiór do 400m. Zasadniczy wpływ na rozchodzenie się tych fal ma atmosfera;
\item krótkie (10 do 100m) - mają własność odbijania się od górnych warstw atmosfery, obejmują dzięki temu swoim zasięgiem całą kulę ziemską;
\item ultrakrótkie (10cm do 10m) - fale rozchodzą się liniowo, dlatego ich zasięg ograniczony jest krzywizną kuli ziemskiej, stosowane w radiostacjach.
\end{itemize}
Fale radiowe ulegają rozproszeniu, pochłanianiu, odbiciu, załamaniu.\\

\textbf{Mikrofale} długości do 30cm są stosowane w radarach, radioteleskopach, urządzeniach grzewnych, łączach telekomunikacyjnych, w astronomii.\\

\textbf{Podsumowując:} fale o dalekim zasięgu najczęsciej uzyskują swe właściwości dzięki zjawisku odbicia, w omawianych przypadkach od górnych warstw atmosfery. Fale o krótkim zasięgu są pochłaniane przez ziemię lub rozchodzą się liniowo, co powoduje ich nieosiągalność tuż przy ziemi po pewnym czasie rozchodzenia (ze względu na okrągłe kształty ziemi).

\end{solution}