\section*{Zadanie 6.}
\begin{task}
Omówić na czym polega efekt naskórkowy i jakie jest jego znaczenie w technice. Jak jest on zależny od częstotliwości? Srebro ma przewodność właściwą ok. $\sigma_{1}=6*10^{7} \big{[}\cfrac{S}{m}\big{]}$ a mosiądz ok. $\sigma_{2}=3*10^{7} \big{[}\cfrac{S}{m}\big{]}$. Zaproponować (i uzasadnić) jak grubą warstwą należy pokryć mosiężny falowód pracujący na częstotliwości 10GHz, aby uzyskać efekt zmniejszenia tłumienia przy stosunkowo małym koszcie.\\
\end{task}

\begin{solution}
\begin{itemize}
\item Efekt naskórkowy polega na tym, że kiedy fala elektromagnetyczna napotyka przewodnik, to wytwarza się na jego powierzchni prąd. Fale nie mogą wnikać w idealny przewodnik. Ładunki i prądy mogą istnieć tylko w nieskończenie cienkiej warstwie na powierzchni idealnego przewodnika.
\item Gdy przewodnik nie jest idealny, mówimy o głębokości wnikania $\delta=\cfrac{1}{\alpha}, \ \ \alpha\approx\sqrt{\frac{\omega\mu\sigma}{2}}$. Im większa częstotliwość, tym mniejsza głębokość wnikania i wtedy metal zachowuje się jak PEC.\\
\item Należy policzyć głębokość wnikania w srebrze przy 10 GHz. Srebro ma większą przewodność od mosiądzu, czyli wprowadza mniejsze tłumienie.
$$\delta_{w}=\cfrac{1}{\sqrt{\mu_{0}\sigma\pi}\sqrt{f}}=0.65 \big{[}\mu m\big{]}$$
\end{itemize}
\end{solution}